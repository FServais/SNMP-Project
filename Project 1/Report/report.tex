\documentclass[a4paper,titlepage]{article}

%PACKAGES
\usepackage[utf8]{inputenc}
\usepackage[T1]{fontenc}
\usepackage[francais]{babel}
\usepackage{amsmath}
\usepackage{amssymb}
\usepackage{mathrsfs}
\usepackage{fancyhdr}
\usepackage{lmodern}
\usepackage{graphicx}
\usepackage{geometry}
\usepackage{fancybox}
\usepackage{textcomp}

%Symbole euro
\usepackage{eurosym}

%Listings : affichage code
\usepackage{listings}


%Elements de la page de garde
\begin{document}

\begin{titlepage}

\begin{figure}
\centering
\includegraphics[width=5cm]{logo-ulg.png}
\end{figure}



\title{
\vspace{0.2cm}
\LARGE{\textbf{Project 1 - Using SNMP tools}} \\ \textsc{Managing and securing computer networks}
\author{\textbf{Floriane Magera} \small{(S111295})\\\textbf{Fabrice Servais} \small{(S111093})}\\
\date{March 10, 2015}
\rule{15cm}{1.5pt}
}

%\geometry{hmargin=2.5cm}
\end{titlepage}

%DOCUMENT
\pagestyle{fancy}
\lhead{Project 1 - Using SNMP tools}
\rhead{Managing and securing computer networks}

%Page de garde
\maketitle

\section{Retrieving variables manually}

	\subsection{What is the system description (sysDescr)?}
We use the \texttt{snmpget} command which will get the value of the leaf "sysDescr" (1.3.6.1.2.1.1.1.0) on the OID tree.
\begin{center}
	\texttt{snmpget hawk.run.montefiore.ulg.ac.be 1.3.6.1.2.1.1.1.0 -v 2c -c run69Zork!}
\end{center}
This is the reply: 
\begin{center}
	\texttt{iso.3.6.1.2.1.1.1.0 = STRING: "Cisco Internetwork Operating System Software 
IOS (tm) C2600 Software (C2600-IK8O3S-M), Version 12.2(11)T,  RELEASE SOFTWARE (fc1)
TAC Support: http://www.cisco.com/tac
Copyright (c) 1986-2002 by cisco Systems, Inc.
Compiled Thu 01-Aug-02 12:47 by cca"}
\end{center}


	\subsection{Is IP forwarding enabled?}
In the same way we used \texttt{snmpget} previously, we get the value of the object "ipForwarding" (1.3.6.1.2.1.4.1.0).
\begin{center}
	\texttt{snmpget hawk.run.montefiore.ulg.ac.be 1.3.6.1.2.1.4.1.0 -v 2c -c run69Zork!}
\end{center}
This is the reply: 
\begin{center}
	\texttt{iso.3.6.1.2.1.4.1.0 = INTEGER: 1}
\end{center}
The value 1 tells that the forwarding is active.\footnote{http://tools.cisco.com/Support/SNMP/do/BrowseOID.do?local=en\&translate=Translate\&objectInput=1.3.6.1.2.1.4.1}

	\subsection{How many interfaces are present in that router?}
\begin{center}
	\texttt{snmpget hawk.run.montefiore.ulg.ac.be 1.3.6.1.2.1.2.1.0 -v 2c -c run69Zork!}
\end{center}
This is the reply: 
\begin{center}
	\texttt{iso.3.6.1.2.1.2.1.0 = INTEGER: 5}
\end{center}
There are so 5 interfaces.

\paragraph{}

To get the informations about each entry, we use the \texttt{snmpwalk} command which will get the subtree from the OID node that is given to it, in our case \texttt{ifTable} (1.3.6.1.2.1.2.2) :
\begin{center}
	\texttt{snmpwalk hawk.run.montefiore.ulg.ac.be 1.3.6.1.2.1.2.2 -v 2c -c run69Zork!}
\end{center}
This is the reply: 
\begin{center}
	\texttt{\\iso.3.6.1.2.1.2.2.1.1.1 = INTEGER: 1
\\iso.3.6.1.2.1.2.2.1.1.2 = INTEGER: 2
\\iso.3.6.1.2.1.2.2.1.1.3 = INTEGER: 3
\\iso.3.6.1.2.1.2.2.1.1.4 = INTEGER: 4
\\iso.3.6.1.2.1.2.2.1.1.5 = INTEGER: 5
\\iso.3.6.1.2.1.2.2.1.2.1 = STRING: "FastEthernet0/0"
\\iso.3.6.1.2.1.2.2.1.2.2 = STRING: "Serial0/0"
\\iso.3.6.1.2.1.2.2.1.2.3 = STRING: "FastEthernet0/1"
\\iso.3.6.1.2.1.2.2.1.2.4 = STRING: "Serial0/1"
\\iso.3.6.1.2.1.2.2.1.2.5 = STRING: "Null0"
\\iso.3.6.1.2.1.2.2.1.3.1 = INTEGER: 6
\\iso.3.6.1.2.1.2.2.1.3.2 = INTEGER: 22
\\iso.3.6.1.2.1.2.2.1.3.3 = INTEGER: 6
\\iso.3.6.1.2.1.2.2.1.3.4 = INTEGER: 22
\\iso.3.6.1.2.1.2.2.1.3.5 = INTEGER: 1
\\iso.3.6.1.2.1.2.2.1.4.1 = INTEGER: 1500
\\iso.3.6.1.2.1.2.2.1.4.2 = INTEGER: 1500
\\iso.3.6.1.2.1.2.2.1.4.3 = INTEGER: 1500
\\iso.3.6.1.2.1.2.2.1.4.4 = INTEGER: 1500
\\iso.3.6.1.2.1.2.2.1.4.5 = INTEGER: 1500
\\iso.3.6.1.2.1.2.2.1.5.1 = Gauge32: 100000000
\\iso.3.6.1.2.1.2.2.1.5.2 = Gauge32: 4000000
\\iso.3.6.1.2.1.2.2.1.5.3 = Gauge32: 100000000
\\iso.3.6.1.2.1.2.2.1.5.4 = Gauge32: 1544000
\\iso.3.6.1.2.1.2.2.1.5.5 = Gauge32: 4294967295
\\iso.3.6.1.2.1.2.2.1.6.1 = Hex-STRING: 00 07 85 A8 83 20 
\\iso.3.6.1.2.1.2.2.1.6.2 = ""
\\iso.3.6.1.2.1.2.2.1.6.3 = Hex-STRING: 00 07 85 A8 83 21 
\\iso.3.6.1.2.1.2.2.1.6.4 = ""
\\iso.3.6.1.2.1.2.2.1.6.5 = ""
\\iso.3.6.1.2.1.2.2.1.7.1 = INTEGER: 1
\\iso.3.6.1.2.1.2.2.1.7.2 = INTEGER: 1
\\iso.3.6.1.2.1.2.2.1.7.3 = INTEGER: 1
\\iso.3.6.1.2.1.2.2.1.7.4 = INTEGER: 2
\\iso.3.6.1.2.1.2.2.1.7.5 = INTEGER: 1
\\iso.3.6.1.2.1.2.2.1.8.1 = INTEGER: 1
\\iso.3.6.1.2.1.2.2.1.8.2 = INTEGER: 1
\\iso.3.6.1.2.1.2.2.1.8.3 = INTEGER: 1
\\iso.3.6.1.2.1.2.2.1.8.4 = INTEGER: 2
\\iso.3.6.1.2.1.2.2.1.8.5 = INTEGER: 1
\\iso.3.6.1.2.1.2.2.1.9.1 = Timeticks: (3227) 0:00:32.27
\\iso.3.6.1.2.1.2.2.1.9.2 = Timeticks: (2928) 0:00:29.28
\\iso.3.6.1.2.1.2.2.1.9.3 = Timeticks: (3228) 0:00:32.28
\\iso.3.6.1.2.1.2.2.1.9.4 = Timeticks: (2769) 0:00:27.69
\\iso.3.6.1.2.1.2.2.1.9.5 = Timeticks: (0) 0:00:00.00
\\iso.3.6.1.2.1.2.2.1.10.1 = Counter32: 273891205
\\iso.3.6.1.2.1.2.2.1.10.2 = Counter32: 52262172
\\iso.3.6.1.2.1.2.2.1.10.3 = Counter32: 76235583
\\iso.3.6.1.2.1.2.2.1.10.4 = Counter32: 0
\\iso.3.6.1.2.1.2.2.1.10.5 = Counter32: 0
\\iso.3.6.1.2.1.2.2.1.11.1 = Counter32: 1684121
\\iso.3.6.1.2.1.2.2.1.11.2 = Counter32: 385624
\\iso.3.6.1.2.1.2.2.1.11.3 = Counter32: 468126
\\iso.3.6.1.2.1.2.2.1.11.4 = Counter32: 0
\\iso.3.6.1.2.1.2.2.1.11.5 = Counter32: 0
\\iso.3.6.1.2.1.2.2.1.12.1 = Counter32: 1869685
\\iso.3.6.1.2.1.2.2.1.12.2 = Counter32: 132715
\\iso.3.6.1.2.1.2.2.1.12.3 = Counter32: 136332
\\iso.3.6.1.2.1.2.2.1.12.4 = Counter32: 0
\\iso.3.6.1.2.1.2.2.1.12.5 = Counter32: 0
\\iso.3.6.1.2.1.2.2.1.13.1 = Counter32: 0
\\iso.3.6.1.2.1.2.2.1.13.2 = Counter32: 0
\\iso.3.6.1.2.1.2.2.1.13.3 = Counter32: 0
\\iso.3.6.1.2.1.2.2.1.13.4 = Counter32: 0
\\iso.3.6.1.2.1.2.2.1.13.5 = Counter32: 0
\\iso.3.6.1.2.1.2.2.1.14.1 = Counter32: 0
\\iso.3.6.1.2.1.2.2.1.14.2 = Counter32: 1
\\iso.3.6.1.2.1.2.2.1.14.3 = Counter32: 0
\\iso.3.6.1.2.1.2.2.1.14.4 = Counter32: 0
\\iso.3.6.1.2.1.2.2.1.14.5 = Counter32: 0
\\iso.3.6.1.2.1.2.2.1.15.1 = Counter32: 0
\\iso.3.6.1.2.1.2.2.1.15.2 = Counter32: 3
\\iso.3.6.1.2.1.2.2.1.15.3 = Counter32: 0
\\iso.3.6.1.2.1.2.2.1.15.4 = Counter32: 0
\\iso.3.6.1.2.1.2.2.1.15.5 = Counter32: 0
\\iso.3.6.1.2.1.2.2.1.16.1 = Counter32: 143253584
\\iso.3.6.1.2.1.2.2.1.16.2 = Counter32: 54688364
\\iso.3.6.1.2.1.2.2.1.16.3 = Counter32: 112095835
\\iso.3.6.1.2.1.2.2.1.16.4 = Counter32: 0
\\iso.3.6.1.2.1.2.2.1.16.5 = Counter32: 0
\\iso.3.6.1.2.1.2.2.1.17.1 = Counter32: 1051186
\\iso.3.6.1.2.1.2.2.1.17.2 = Counter32: 702708
\\iso.3.6.1.2.1.2.2.1.17.3 = Counter32: 623349
\\iso.3.6.1.2.1.2.2.1.17.4 = Counter32: 0
\\iso.3.6.1.2.1.2.2.1.17.5 = Counter32: 0
\\iso.3.6.1.2.1.2.2.1.18.1 = Counter32: 272692
\\iso.3.6.1.2.1.2.2.1.18.2 = Counter32: 0
\\iso.3.6.1.2.1.2.2.1.18.3 = Counter32: 443574
\\iso.3.6.1.2.1.2.2.1.18.4 = Counter32: 0
\\iso.3.6.1.2.1.2.2.1.18.5 = Counter32: 0
\\iso.3.6.1.2.1.2.2.1.19.1 = Counter32: 0
\\iso.3.6.1.2.1.2.2.1.19.2 = Counter32: 0
\\iso.3.6.1.2.1.2.2.1.19.3 = Counter32: 0
\\iso.3.6.1.2.1.2.2.1.19.4 = Counter32: 0
\\iso.3.6.1.2.1.2.2.1.19.5 = Counter32: 0
\\iso.3.6.1.2.1.2.2.1.20.1 = Counter32: 0
\\iso.3.6.1.2.1.2.2.1.20.2 = Counter32: 0
\\iso.3.6.1.2.1.2.2.1.20.3 = Counter32: 0
\\iso.3.6.1.2.1.2.2.1.20.4 = Counter32: 0
\\iso.3.6.1.2.1.2.2.1.20.5 = Counter32: 0
\\iso.3.6.1.2.1.2.2.1.21.1 = Gauge32: 0
\\iso.3.6.1.2.1.2.2.1.21.2 = Gauge32: 0
\\iso.3.6.1.2.1.2.2.1.21.3 = Gauge32: 0
\\iso.3.6.1.2.1.2.2.1.21.4 = Gauge32: 0
\\iso.3.6.1.2.1.2.2.1.21.5 = Gauge32: 0
\\iso.3.6.1.2.1.2.2.1.22.1 = OID: ccitt.0
\\iso.3.6.1.2.1.2.2.1.22.2 = OID: ccitt.0
\\iso.3.6.1.2.1.2.2.1.22.3 = OID: ccitt.0
\\iso.3.6.1.2.1.2.2.1.22.4 = OID: ccitt.0
\\iso.3.6.1.2.1.2.2.1.22.5 = OID: ccitt.0}
\end{center}

	\subsection{What is the MAC address of FastEthernet0/0?}
We saw that "FastEthernet0/0" is the number 1 entry in the table. Knowing that the MAC address is contained in "ifPhysAddress" (OID: 1.3.6.1.2.1.2.2.1.6), we can access the information using: 
\begin{center}
	\texttt{snmpget hawk.run.montefiore.ulg.ac.be 1.3.6.1.2.1.2.2.1.6.1 -v 2c -c run69Zork!}
\end{center}
This is the reply: 
\begin{center}
	\texttt{iso.3.6.1.2.1.2.2.1.6.1 = Hex-STRING: 00 07 85 A8 83 20 }
\end{center}

	\subsection{What is the MTU of Serial0/1?}
We saw that "Serial0/1" is the number 4 entry in the table. Knowing that the MTU is contained in "ifMtu" (OID: 1.3.6.1.2.1.2.2.1.4), we can access the information using: 
\begin{center}
	\texttt{snmpget hawk.run.montefiore.ulg.ac.be 1.3.6.1.2.1.2.2.1.4.1 -v 2c -c run69Zork!}
\end{center}
This is the reply: 
\begin{center}
	\texttt{iso.3.6.1.2.1.2.2.1.4.1 = INTEGER: 1500}
\end{center}

	\subsection{What is the default IP TTL?}
The information is contained in the "ipDefaultTTL" (OID: 1.3.6.1.2.1.4.2). 

\begin{center}
	\texttt{snmpget hawk.run.montefiore.ulg.ac.be 1.3.6.1.2.1.4.2.0 -v 2c -c run69Zork!}
\end{center}
\begin{center}
	\texttt{iso.3.6.1.2.1.4.2.0 = INTEGER: 255}
\end{center}
The TTL is so of 255.

	\subsection{Set the default IP TTL to a different value}


\section{Retrieving variables from a script}


\newpage

\end{document}