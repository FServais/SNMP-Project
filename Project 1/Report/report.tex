\documentclass[a4paper,titlepage]{article}

%PACKAGES
\usepackage[utf8]{inputenc}
\usepackage[T1]{fontenc}
\usepackage[english]{babel}
\usepackage{amsmath}
\usepackage{amssymb}
\usepackage{mathrsfs}
\usepackage{fancyhdr}
\usepackage{lmodern}
\usepackage{graphicx}
\usepackage{geometry}
\usepackage{fancybox}
\usepackage{textcomp}

%Symbole euro
\usepackage{eurosym}

%Listings : affichage code
\usepackage{listings}


%Elements de la page de garde
\begin{document}

\begin{titlepage}

\begin{figure}
\centering
\includegraphics[width=5cm]{logo-ulg.png}
\end{figure}



\title{
\vspace{0.2cm}
\LARGE{\textbf{Project 1 - Using SNMP tools}} \\ \textsc{Managing and securing computer networks}
\author{\textbf{Floriane Magera} \small{(S111295})\\\textbf{Fabrice Servais} \small{(S111093})}\\
\date{March 10, 2015}
\rule{15cm}{1.5pt}
}

%\geometry{hmargin=2.5cm}
\end{titlepage}

%DOCUMENT
\pagestyle{fancy}
\lhead{Project 1 - Using SNMP tools}
\rhead{Managing and securing computer networks}

%Page de garde
\maketitle

\section{Retrieving variables manually}
The relevant MIB files according to the questions are the following:
\begin{itemize}
	\item \texttt{SNMPv2-MIB} (\texttt{sysDescr})
	\item \texttt{IP-MIB} (\texttt{ipForwarding}, \texttt{ipDefaultTTL})
	\item \texttt{IF-MIB} (\texttt{ifNumber}, \texttt{ifDescr}, \texttt{ifType}, \texttt{ifOperStatus}, \texttt{ifPhysAddress}, \texttt{ifMtu})
\end{itemize}

Those three MIB's has been added in the \texttt{~/.snmp/snmp.conf} file.

	\subsection{What is the system description (sysDescr)?}
We use the \texttt{snmpget} command which will get the value of the leaf \texttt{sysDescr} (1.3.6.1.2.1.1.1.0) on the OID tree.
\begin{center}
	\texttt{snmpget hawk.run.montefiore.ulg.ac.be sysDescr.0 -v 2c -c run69Zork!}
\end{center}
This is the reply: 
\begin{center}
	\texttt{SNMPv2-MIB::sysDescr.0 = STRING: Cisco Internetwork Operating System Software 
IOS (tm) C2600 Software (C2600-IK8O3S-M), Version 12.2(11)T,  RELEASE SOFTWARE (fc1)
TAC Support: http://www.cisco.com/tac
Copyright (c) 1986-2002 by cisco Systems, Inc.
Compiled Thu 01-Aug-02 12:47 by cca"}
\end{center}


	\subsection{Is IP forwarding enabled?}
In the same way we used \texttt{snmpget} previously, we get the value of the object \texttt{ipForwarding} (1.3.6.1.2.1.4.1.0).
\begin{center}
	\texttt{snmpget hawk.run.montefiore.ulg.ac.be ipForwarding.0 -v 2c -c run69Zork!}
\end{center}
This is the reply: 
\begin{center}
	\texttt{IP-MIB::ipForwarding.0 = INTEGER: forwarding(1)}
\end{center}
The value is set to "forwarding".

	\subsection{How many interfaces are present in that router?}
\begin{center}
	\texttt{snmpget hawk.run.montefiore.ulg.ac.be ifNumber.0 -v 2c -c run69Zork!}
\end{center}
This is the reply: 
\begin{center}
	\texttt{IF-MIB::ifNumber.0 = INTEGER: 5}
\end{center}
There are so 5 interfaces.

\paragraph{}

To get the informations about each entry, we use the \texttt{snmpwalk} command which will get the subtree from the OID node that is given to it, in our case \texttt{ifTable} (1.3.6.1.2.1.2.2) :
\begin{center}
	\texttt{snmpwalk hawk.run.montefiore.ulg.ac.be ifTable -v 2c -c run69Zork!}
\end{center}
We got all the informations contained in that table, here are the useful ones : 
\begin{verbatim}
	IF-MIB::ifIndex.1 = INTEGER: 1
	IF-MIB::ifIndex.2 = INTEGER: 2
	IF-MIB::ifIndex.3 = INTEGER: 3
	IF-MIB::ifIndex.4 = INTEGER: 4
	IF-MIB::ifIndex.5 = INTEGER: 5
	IF-MIB::ifDescr.1 = STRING: FastEthernet0/0
	IF-MIB::ifDescr.2 = STRING: Serial0/0
	IF-MIB::ifDescr.3 = STRING: FastEthernet0/1
	IF-MIB::ifDescr.4 = STRING: Serial0/1
	IF-MIB::ifDescr.5 = STRING: Null0
	IF-MIB::ifType.1 = INTEGER: ethernetCsmacd(6)
	IF-MIB::ifType.2 = INTEGER: propPointToPointSerial(22)
	IF-MIB::ifType.3 = INTEGER: ethernetCsmacd(6)
	IF-MIB::ifType.4 = INTEGER: propPointToPointSerial(22)
	IF-MIB::ifType.5 = INTEGER: other(1)
	IF-MIB::ifMtu.1 = INTEGER: 1500
	IF-MIB::ifMtu.2 = INTEGER: 1500
	IF-MIB::ifMtu.3 = INTEGER: 1500
	IF-MIB::ifMtu.4 = INTEGER: 1500
	IF-MIB::ifMtu.5 = INTEGER: 1500
	IF-MIB::ifPhysAddress.1 = STRING: 0:7:85:a8:83:20
	IF-MIB::ifPhysAddress.2 = STRING: 
	IF-MIB::ifPhysAddress.3 = STRING: 0:7:85:a8:83:21
	IF-MIB::ifPhysAddress.4 = STRING: 
	IF-MIB::ifPhysAddress.5 = STRING: 
	IF-MIB::ifOperStatus.1 = INTEGER: up(1)
	IF-MIB::ifOperStatus.2 = INTEGER: up(1)
	IF-MIB::ifOperStatus.3 = INTEGER: up(1)
	IF-MIB::ifOperStatus.4 = INTEGER: down(2)
	IF-MIB::ifOperStatus.5 = INTEGER: up(1)
\end{verbatim}
For each interface, we have its index (\texttt{IF-MIB::ifIndex}), its name (\texttt{IF-MIB::ifDescr}), its type (\texttt{IF-MIB::ifType}) and its status (\texttt{IF-MIB::ifOperStatus}). For the latter, the value is set to "up" when the interface is on and "down" when the interface is off.

	\subsection{What is the MAC address of FastEthernet0/0?}
We saw that "FastEthernet0/0" is at the index 1 in the table. Knowing that the MAC address is contained in \texttt{ifPhysAddress} (OID: 1.3.6.1.2.1.2.2.1.6), we can access the information using: 
\begin{center}
	\texttt{snmpget hawk.run.montefiore.ulg.ac.be ifPhysAddress.1 -v 2c -c run69Zork!}
\end{center}
This is the reply: 
\begin{center}
	\texttt{IF-MIB::ifPhysAddress.1 = STRING: 0:7:85:a8:83:20}
\end{center}

	\subsection{What is the MTU of Serial0/1?}
We saw that "Serial0/1" is at index 4 in the table. Knowing that the MTU is contained in \texttt{ifMtu} (OID: 1.3.6.1.2.1.2.2.1.4), we can access the information using: 
\begin{center}
	\texttt{snmpget hawk.run.montefiore.ulg.ac.be ifMtu.4 -v 2c -c run69Zork!}
\end{center}
This is the reply: 
\begin{center}
	\texttt{IF-MIB::ifMtu.4 = INTEGER: 1500}
\end{center}

	\subsection{What is the default IP TTL?}
The information is contained in the \texttt{ipDefaultTTL} (OID: 1.3.6.1.2.1.4.2). 

\begin{center}
	\texttt{snmpget hawk.run.montefiore.ulg.ac.be ipDefaultTTL.0 -v 2c -c run69Zork!}
\end{center}
\begin{center}
	\texttt{IP-MIB::ipDefaultTTL.0 = INTEGER: 255}
\end{center}
The TTL is so of 255.

	\subsection{Set the default IP TTL to a different value}
We use the \texttt{snmpset} command to try to set a value into the \texttt{ipDefaultTTL} field.
\begin{center}
	\texttt{snmpset -v 2c -c run69Zork! hawk.run.montefiore.ulg.ac.be ipDefaultTTL.0 i 200}
\end{center}
This is the reply:
\begin{verbatim}
	Error in packet.
	Reason: noAccess
	Failed object: IP-MIB::ipDefaultTTL.0
	zsh: exit 2     snmpset -v 2c -c run69Zork! hawk.run.montefiore.ulg.ac.be ipDefaultTTL.0 i 20
\end{verbatim}
We have thus no write access to the object. Since the MIB access policy for that object is "read-write", it means that the SNMP mode access (related to the community) is set to "READ-ONLY".

\section{Retrieving variables from a script}


\newpage

\end{document}