\documentclass[a4paper,titlepage]{article}

%PACKAGES
\usepackage[utf8]{inputenc}
\usepackage[T1]{fontenc}
\usepackage[english]{babel}
\usepackage{amsmath}
\usepackage{amssymb}
\usepackage{mathrsfs}
\usepackage{fancyhdr}
\usepackage{lmodern}
\usepackage{graphicx}
\usepackage{geometry}
\usepackage{fancybox}
\usepackage{textcomp}

%Symbole euro
\usepackage{eurosym}

%Listings : affichage code
\usepackage{listings}


%Elements de la page de garde
\begin{document}

\begin{titlepage}

\begin{figure}
\centering
\includegraphics[width=5cm]{logo-ulg.png}
\end{figure}



\title{
\vspace{0.2cm}
\LARGE{\textbf{Project 2 - Developing a web-based SNMP browser : Detecting the SNMP Agents}} \\ \textsc{Managing and securing computer networks}
\author{\textbf{Floriane Magera} \small{(S111295})\\\textbf{Fabrice Servais} \small{(S111093})}\\
\date{March 10, 2015}
\rule{15cm}{1.5pt}
}

%\geometry{hmargin=2.5cm}
\end{titlepage}

%DOCUMENT
\pagestyle{fancy}
\lhead{Project 2 - Detecting the SNMP Agents}
\rhead{Managing and securing computer networks}

%Page de garde
\maketitle


\section{Introduction}

The part of the project aims at generate a file containing the list of agents located in the network given in "\texttt{config.txt}".
\paragraph{}
The script can be run through a CRON job, using the command :
\begin{center}
	\texttt{crobtab cron.txt}
\end{center}
The \texttt{cron.txt} file contains the cron configuration to run the python script every 15 minutes. This script can be run, for testing purpose, using the command : 
\begin{center}
	python detection.py
\end{center}


\section{Detecting the agents}
First we generated a list of all the targets possible : for each ip adress in the domain, we consider all the credentials possible.  Then depending on the SNMP version of the target, we use different methods to send the request aiming the agent. 
	
	\subsection{SNMPv1 and SNMPv2 (discoverTargets(targets))}
	We used the library pysnmp to send asynchronous requests to all the agents in the domain. Instead of using a snmpget request, we use a getnext request. The agent returns the next available object in its tree. Every target has an entry in a hashtable, with an entry of the form <id of the request, configuration>. When the request is completed, a callback function is called. It will receive the information about that request so we can analyze it :
	\begin{itemize}
	 	\item if we receive an error, the entry corresponding to the target is deleted,
	 	\item otherwise, we do not remove the entry. At the end of the process, the hashtable contains only the agents present in the network.
	 \end{itemize}  


	\subsection{SNMPv3 (\texttt{discoverTargetsV3(targets, numberOfThreads)})}
	As the asynchronous requests did not work with the library we use, we decided to launch threads and use synchronous requests of the library. 
	For that, we use a threadpool (class \texttt{ThreadPool} containing classes of \texttt{Worker}) of 15 threads. We also use the same request as in the previous case. We tried to use a suitable number of threads to make the detection as fast as possible and to avoid overloading the machine or the network.


\section{Results}
We write all the informations about the agents in a XML file. The script takes more or less 2:45 minutes, which seems acceptable regarding to the tradeoff we had to do, facing the SNMPv3 asynchronous requests problem.

\end{document}
